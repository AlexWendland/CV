\documentclass[a4paper,10pt]{article}

%A Few Useful Packages
\usepackage{marvosym}
\usepackage{fontspec} 					%for loading fonts
\usepackage{xunicode,xltxtra,url,parskip} 	%other packages for formatting
\RequirePackage{color,graphicx}
\usepackage[usenames,dvipsnames]{xcolor}
\usepackage[left=1.5cm, right=1.5cm, top=1.3cm, bottom=1.3cm]{geometry}     %Page layout
\usepackage{supertabular} 				%for Grades
\usepackage{titlesec}					%custom \section

%Setup hyperref package, and colours for links
\usepackage{hyperref}
\definecolor{linkcolour}{rgb}{0,0.2,0.6}
\hypersetup{colorlinks,breaklinks,urlcolor=linkcolour, linkcolor=linkcolour}

%FONTS
\defaultfontfeatures{Mapping=tex-text}
%\setmainfont[SmallCapsFont = Fontin SmallCaps]{Fontin}
%%% modified for Karol Kozioł for ShareLaTeX use
\setmainfont[
SmallCapsFont = Fontin-SmallCaps.otf,
BoldFont = Fontin-Bold.otf,
ItalicFont = Fontin-Italic.otf
]
{Fontin.otf}
%%%

%CV Sections inspired by: 
%http://stefano.italians.nl/archives/26
\titleformat{\section}{\Large\scshape\raggedright}{}{0em}{}[\titlerule]
\titlespacing{\section}{0pt}{3pt}{3pt}
%Tweak a bit the top margin
%\addtolength{\voffset}{-1.3cm}

%Italian hyphenation for the word: ''corporations''
\hyphenation{im-pre-se}

%-------------WATERMARK TEST [**not part of a CV**]---------------
\usepackage[absolute]{textpos}

\setlength{\TPHorizModule}{30mm}
\setlength{\TPVertModule}{\TPHorizModule}
\textblockorigin{2mm}{0.65\paperheight}
\setlength{\parindent}{0pt}

%--------------------BEGIN DOCUMENT----------------------
\begin{document}

%WATERMARK TEST [**not part of a CV**]---------------
%\font\wm=''Baskerville:color=787878'' at 8pt
%\font\wmweb=''Baskerville:color=FF1493'' at 8pt
%{\wm 
%	\begin{textblock}{1}(0,0)
%		\rotatebox{-90}{\parbox{500mm}{
%			Typeset by Alessandro Plasmati with \XeTeX\  \today\ for 
%			{\wmweb \href{http://www.aleplasmati.comuv.com}{aleplasmati.comuv.com}}
%		}
%	}
%	\end{textblock}
%}

\pagestyle{empty} % non-numbered pages

\font\fb=''[cmr10]'' %for use with \LaTeX command

%--------------------TITLE-------------
\par{\centering
		{\Huge Alex Wendland
	}\bigskip\par}
\begin{center}
	\textsc{Phone:} +44 7854033626 \hspace{5cm} \textsc{E-mail:} \href{mailto:alex@wendland.org.uk}{\texttt{alex@wendland.org.uk}}\\
	\textsc{Website:} \href{https://warwick.ac.uk/fac/sci/maths/people/staff/wendland/}{https://warwick.ac.uk/fac/sci/maths/people/staff/wendland/}
\end{center}

%--------------------SECTIONS-----------------------------------

%Section: Education
\section{Education}
\begin{tabular}{rl}	
\textsc{} 2015--Current \hspace{-0.12cm} & Preparing thesis for PhD in \textsc{Mathematics}, \textbf{University of Warwick}, Coventry\\
& ``Group structures within graphs'' supervised by Agelos Georgakopoulos\\
&\\
%& 110/110 \small\emph{summa cum laude} | Major: Quantitative Finance\\
%& Thesis: ``Sublinear and Locally Sublinear Prices'' | \small Advisor: Prof. Erio \textsc{Castagnoli}\\
%&\normalsize \textsc{Gpa}: 28.61/30\hyperlink{grds}{\hfill | \footnotesize Detailed List of Exams}\\
\textsc{} 2011--2015& Masters in \textsc{Mathematics}, \textbf{University of Warwick}, Coventry\\
& 1\textsuperscript{st} Class (Hons)\\ &\\
\textsc{} 2004--2011& Secondary school, \textbf{Beechen Cliff School}, Bath\\
&  A Levels: Further Maths ($A^{\ast}$), Maths ($A^{\ast}$), Physics ($A$)\\
\end{tabular}

\section{Employment History}
\begin{tabular}{p{2.25cm}|p{15cm}}
	\textsc{2019--Current} & Centre for Science and Policy -- \textsc{University of Cambridge}\\&\footnotesize{I am currently on a 3-month EPSRC-funded Policy Internship at the Centre for Science and Policy (CSaP). CSaP's main mission is to connect policy makers with Academics to encourage science-based approaches to policy. My main project at CSaP is to use their database with Natural Language Processing to help them find experts with the correct skills. I also attend workshops and seminars on public policy, I find marrying the science with practical solutions fascinating.}\\\multicolumn{2}{c}{} \\
	\textsc{2015--2019} & Postgraduate Teacher -- \textsc{University of Warwick}\\&\footnotesize{I have a wide background in teaching, please see my webpage for an exact listing of jobs. Roles include; class teacher at multiple levels (1st year to 4th) of undergraduate mathematics and computer science; supervising 1st years on a one on five setting; giving revision lectures to up to 400 attendees; one on one sessions with students with special learning difficulties. Teaching has one of the richest forms of personal and professional development I have experienced. Learning how different people think and view problems has given me countless insights into how people behave when challenged.}\\\multicolumn{2}{c}{} \\
	\textsc{2014--2019} & Assorted short roles -- \textsc{University of Warwick, Mathematics department}\\&\footnotesize{I have carried out various short roles for the Mathematics department. I helped run open days, organising a team of other workers. I worked as a videographer for multiple conferences at Warwick, recording and editing. I have invigilated and marked exams, in one instance covering for an ill Academic as the sole marker.}\\\multicolumn{2}{c}{} \\
	\textsc{2009--2011}& Team member / Retail supervisor -- \textsc{Odeon Cinemas}, Bath\\&\footnotesize{During sixth form, I worked in a large team responsible for the operations of a cinema. Within my first year I was promoted to a retail supervisor in charge of the customer retail team and managing stock level. }\\\multicolumn{2}{c}{} \\
\end{tabular}

\section{Academic Roles}
\begin{tabular}{p{2.25cm}|p{15cm}}
	\textsc{2011--current}& Student Staff Liaison Committee (Chair 2013--15 and 2017--18) -- \textsc{University of Warwick}\\&\footnotesize{I am an active member of the Staff and Student Liaison Committee for the Warwick Maths department. I joined in my first year of undergraduate and have been a member since. I was chair from 2013 to 2015 then joint chair from 2017 to 2018. Here I honed my communication and conflict resolution skills when discussing problems people were facing and presenting them to the department in a constructive manner.}\\\multicolumn{2}{c}{} \\
	\textsc{2012--Current} & Talks\\&\footnotesize{I have given many talks throughout my academic life in various locations including Cambridge, Glasgow, Iceland, Imperial College, Oxford, Warwick. The first conference talk I gave was in 2013 at the Mathematician of Tomorrow conference. A highlight was delivering the Mathematical Colloquium at the University of Iceland in February 2018. Please see my webpage for a selected list of talks.}\\\multicolumn{2}{c}{}\\
	\textsc{2013--Current} & Conferences \\&\footnotesize{I have enjoyed attending conferences in various countries. Communicating and learning mathematics with people from different backgrounds has enhanced my PhD experience. A highlight was attending the European Study Group with Industry conference in 2018, where I worked with a team of Mathematicians on a scheduling problem provided by Heathrow Airport. Applying mathematics to real world data was of particular interest.}\\\multicolumn{2}{c}{} \\
	\textsc{2014--Current} & Academic working visits \\&\footnotesize{I have visited and worked with academics at various institutions in the Czech Republic, Iceland and the UK, learning to work with different teams in a variety of environments.}\\\multicolumn{2}{c}{} \\
\end{tabular}
\begin{tabular}{p{2.25cm}|p{15cm}}
	\textsc{2016--Current}& Diversity Committee -- \textsc{University of Warwick, Mathematics department}\\&\footnotesize{I am a member of the diversity committee as a Postgraduate representative. I have contributed to the successful submission for a Bronze Athena Swan award for the Mathematics department in 2017..}\\\multicolumn{2}{c}{} \\
	\textsc{2016--2017}& Postgraduate Seminar Organiser -- \textsc{University of Warwick, Mathematics department}\\&\footnotesize{I organised the Postgraduate Seminar for this Academic year, which entailed a years’ programme of talks as well as postgraduate talks for two open days. This developed my organisation skills, and gave me experience of acting under time pressure when speakers pulled out.}\\\multicolumn{2}{c}{} \\
	\textsc{2013--14}& Academic Support for Warwick Maths Society -- \textsc{University of Warwick}\\&\footnotesize{I was the Academic Support for Warwick Maths Society where I organised many revision lectures given by others, and myself. I ran weekly drop in sessions, LaTeX support classes, lecture catch up sessions and helped with other activities run by the society such as socials, academic talks and career events.}\\\multicolumn{2}{c}{} \\
\end{tabular}

\section{Awards and Grants}
\begin{tabular}{p{2.25cm}|p{15cm}}
	\textsc{2015--Current}& Engineering and Physical Sciences Research Council Doctoral Award -- \textsc{University of Warwick}\\&\footnotesize{Funding to undertake a PhD in Mathematics at the University of Warwick.}\\\multicolumn{2}{c}{} \\
	\textsc{2016--17}& UKRI Policy Internship -- \textsc{EPSRC}\\&\footnotesize{Awarded 3 months of additional PhD funding and £2,400 travel/living costs to undertake an internship at the Centre for Science and Policy at the University of Cambridge.}\\\multicolumn{2}{c}{} \\
	\textsc{2016--17}& Mathematics Institute Postgraduate Teaching Prize -- \textsc{University of Warwick}\\&\footnotesize{``These are awarded to supervisors/TAs by vote of the undergraduates, so are earned through the genuine appreciation of people you taught in 2016-17.”}\\\multicolumn{2}{c}{} \\
	\textsc{2013--14}& Undergraduate Research Scholarship Scheme -- \textsc{University of Warwick}\\&\footnotesize{Awarded £1000 for a summer project with Dr Krzysztof Latuszynski in University of Warwick in 2013. Awarded £1000 for a summer project with Professor Daniel Král' in University of Warwick, Charles University and the University of West Bohemia in 2014, which was also supported by Graph coloring and structure grant of the Czech Science Foundation.}\\\multicolumn{2}{c}{} \\
	\textsc{2013--14}& Giving to Warwick Student Prize -- \textsc{University of Warwick}\\&\footnotesize{``In recognition of outstanding contribution to the department of Mathematics”}\\\multicolumn{2}{c}{} \\
	\textsc{2013}& Summer of student innovation -- \textsc{JISC}, Bristol\\&\footnotesize{Award £5000 to develop a virtual learning platform for university students teaching them graduate level techniques.}\\\multicolumn{2}{c}{} \\
	\textsc{2012--13}& Ron Lockhart Student Prize -- \textsc{University of Warwick}\\&\footnotesize{``In recognition of Academic excellence and outstanding contribution to the Mathematics Institute."}\\\multicolumn{2}{c}{} \\
\end{tabular}

\section{Programming Projects}
\begin{tabular}{rl}
	Intermediate Knowledge:&  \textsc{LaTeX}, Python and Java\\
	Basic Knowledge:& Matlab, \textsc{MAGMA}, C, SQL and Lua\\
	GitHub:& \href{https://github.com/AlexWendland}{https://github.com/AlexWendland}
\end{tabular}\\
\vspace{0.1 in}\\
\begin{tabular}{p{2.25cm}|p{15cm}}
	\textsc{2019--Current} & EPSRC-funded Policy Internship -- \textsc{Centre for Science and Policy, University of Cambridge}\\&\footnotesize{Working on a Natural Language Processing project on the profiles of experts to better enable search methods. These tools are also being used to understand how different areas of policy draw upon the same pool of experts to better inform cross governmental cooperation. For this project I have been given access to CSaP's database, processing data using Python and SQL. Please find a depositary on my GitHub with code to date.}\\\multicolumn{2}{c}{} \\
	\textsc{Summer 2018}& Droplet size analysis -- \textsc{University of Warwick, Warwick Manufacturing Group}\\&\footnotesize{For the paper ``The Effects of Height and Blow Time on Droplet-Size Distributions in the Emulsion Phase of a BOF'' I used MatLab to analyse the distribution of gas droplet radii within a steel mixing furnace. I deployed statistical techniques such as chi-squared tests and data visualisation to give insights into the mixing process.}\\\multicolumn{2}{c}{} \\
	\textsc{Summer 2018} & European Study Group with Industry -- \textsc{University of Bath} \\&\footnotesize{I attended a weeklong conference where a team of Mathematicians worked on a problem set by Heathrow Airport. The question was ``is it cost effective to schedule construction work?’’ For this we analysed 5 years of project data provided by Heathrow. We designed a data driven statistical model in Python and trialled out the cost effectiveness of different scheduling strategies. Please find a selection of the code on my GitHub account.}\\\multicolumn{2}{c}{} \\
\end{tabular}
\begin{tabular}{p{2.25cm}|p{15cm}}
	\textsc{Summer 2013} & Summer of Student Innovation -- \textsc{JISC, London} \\&\footnotesize{Three students and I won £5000 of funding to develop a Virtual Learning Environment aimed at non-Maths university students to learn university level Mathematics. The USP of the project was a question generator that used decision trees to understand where students made mistakes. This was called Mimir and the depositary is available on my GitHub.}\\\multicolumn{2}{c}{} \\
	\textsc{2012--2013} & Hypochondriapp -- \textsc{University of Warwick, Computer Science department} \\&\footnotesize{Two students and I programmed an Android App with a Java server back end as part of the Group Programming Project during the 2nd year of my undergraduate degree. The server used twitter data to access the density of `illness' within areas of London, then used Transport for London data with a cellular automaton styled model to predict the future spread of flu. Please find the code on my GitHub account.}\\\multicolumn{2}{c}{} \\
\end{tabular}

\section{Publications and Preprints}
\footnotesize{I have worked on papers in varying areas, some in pure Mathematics (graph theory and higher dimensional algebra), Statistics and the Warwick Manufacturing Group.} 
\begin{itemize}
	\item[--] 
	S. Spooner, A. Wendland, Z. Li, S. Sridhar, M. A. Williams and J. M. Warnett \emph{The Effects of Height and Blow Time on Droplet-Size-Distributions in the Emulsion Phase of a BOF} Available on request, to be submitted.
	\item[--] 
	K. Latuszynsky, G. Morina and A. Wendland. \emph{From the Bernoulli Factory to a Dice Enterprise via Perfect Sampling of Markov Chains.} Available on request, to be submitted.
	\item[--]
	D. Rumynin and A. Wendland. \emph{2-Groups, 2-Characters and Burnside Rings.} \href{https://www.sciencedirect.com/science/article/pii/S0001870818303347}{Advances in Mathematics}, 338: 196-236, 2018.
	\item[--]
	A. Wendland. \emph{Coloring of Plane Graphs with Unique Maximal Colors on Faces}. \href{http://onlinelibrary.wiley.com/doi/10.1002/jgt.22002/abstract}{J. Graph Theory}, 83(4):359–371, 2016.
\end{itemize}

\section{Interests and Activities}
I enjoy cycling, hiking, playing board games and politics. I take an interest in genetic algorithms and neural networks, programming in my free time with a particular focus on how it is being used and effects upon society. Recently I have been reading a lot on the topic of Behavioural Science and attending seminars on its impact in policy.\\

\end{document}
